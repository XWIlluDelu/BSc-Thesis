\begin{enabstract}{Continued Fractions; Kuzmin's Theorem; Gauss Measure; Mixing Property; Ergodicity}

This paper mainly discusses the strong mixing property and ergodicity of the continued fraction dynamical system, and thus obtains the asymptotic independence, $\psi$-mixing, and some limit properties of the partial quotient sequence.

Chapter 1 provides a brief introduction to the research history and current state of continued fractions.

Chapter 2 expounds on the definition and fundamental properties of continued fractions, establishes a convergence test of continued fractions, and thereby demonstrates the convergence of regular continued fractions. Subsequently, the basic principles of the metric theory of continued fractions, the Gauss transformation and Gauss measure are introduced. It is demonstrated that the Gauss transformation preserves the Gauss measure and that the Gauss measure is equivalent to the Lebesgue measure. Additionally, the strict stationarity, reversibility, and non-independence of the partial quotient sequence of continued fractions are proved.

Chapter 3 introduces measure-preserving transformations, semi-algebra, algebra, and monotonic classes, then provides a method for verifying measure-preserving transformations. Subsequently, using Kuzmin's theorem, a solution to the Gauss problem is given, and the strong mixing of the continued fraction dynamical system, as well as the $\psi-$mixing and asymptotic independence of the partial quotient sequence, are proven.

Chapter 4 introduces ergodic systems and the fundamental theorem of ergodic theory - Birkhoff's ergodic theorem. After that, by proving that strong mixing implies ergodicity and ergodicity implies weak mixing, it proves the ergodicity of the continued fraction dynamical system. Additionally, by applying Birkhoff's ergodic theorem three properties of the partial quotient sequence are proved.

\end{enabstract}
