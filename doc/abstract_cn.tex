\begin{cnabstract}{连分数;Kuzmin定理;Gauss测度;混合性;遍历性}
    \hspace{2em}
    本文主要讨论了连分数动力系统的强混合性与遍历性,并由此得到了部分商序列的渐进独立性, $\psi$-混合性以及一些极限性质.\par
    第一章简单介绍了连分数的研究历史和现状.\par
    第二章介绍了连分数的定义和基本性质,得到了连分数收敛的判定条件,从而说明了正则连分数的收敛性.之后介绍了连分数度量理论的基本内容,并研究了Gauss变换和Gauss测度,说明了Gauss 变换保持 Gauss 测度以及 Gauss 测度与 Lebesgue 测度的等价性,连分数部分商序列的严平稳、可逆和不独立性.\par
    第三章介绍了保测变换和半代数、代数、单调类,并给出了保测变换的验证方法.之后,通过Kuzmin定理给出了Gauss问题的一个解以及证明了连分数动力系统的强混合性与部分商序列的$\psi$-混合性、渐进独立性.\par
    第四章介绍了遍历系统和遍历论的基本定理——Birkhoff遍历定理.之后,证明了强混合性蕴含弱混合性蕴含遍历性,并由此证明了连分数动力系统的遍历性.并且运用Birkhoff遍历定理证明了部分商序列的三个性质.

\end{cnabstract}