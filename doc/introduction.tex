\seccontent
\section{绪论}
\hspace{12em}

将形如
\begin{equation}\label{definition}
a_0+\frac{1}{a_1+\dfrac{1}{a_2+\cdots+\dfrac{1}{a_n+\cdots}}}
\end{equation}
的式子称为连分数.关于连分数的研究历史悠久,在古希腊数学家Euclid计算两个正整数$a,b$的最大公因数的辗转相除法中,实际上已经得出了有理数$\dfrac{a}{b}$的连分数表示.在另外两个古老的问题:平方根估计和丢番图方程$ax+by=c$求解中,早期的数学家们也在某种意义上使用了连分数.在这之后,意大利数学家Fibonacci首次尝试对连分数进行一般定义,而Bombelli被认为是真正使用了现代连分数理论的第一位数学家.到了18世纪, Euler, Lambert和Lagrange等数学家为连分数理论做出了重大的贡献.在19世纪初, Gauss提出的一个问题让连分数从纯粹的数论研究领域变成结合现代分析的研究领域.这个问题出现在Gauss写给Laplace的信中, Gauss想要估计
$$m_n(x)-\frac{\ln(1+x)}{\ln{2}}$$
Gauss在信中声称他已经证明了$\lim\limits_{n\to\infty}m_n(x)=\dfrac{\ln(1+x)}{\ln{2}},\quad(0\leqslant x\leqslant 1)$,其中
$m_n(x)=\lambda({u\in[0,1)\mid\tau^n(u)\leqslant x})$, $\lambda$为Lebesgue测度, $\tau$为Gauss变换.\par
一个世纪后, Kuzmin给出了上面问题的一个估计\textsuperscript{\cite{Kuzmin}}:
$$m_n\left(x\right)=\frac{\ln{\left(1+x\right)}}{\ln{2}}+r_n\left(x\right),\quad r_n\left(x\right)=O\left(q^{\sqrt n}\right).$$\par
Levy用概率语言给出了更精细的估计\textsuperscript{\cite{Iosifescu,Levy}}:
$$r_n(x)=O(q^n).$$\par
从此,对这个问题以及相关问题的研究得到高速发展,详情请见参考文献\cite{Liu_Peng,The_long_history,Iosifescu,Gauss-Kuzmin_type,Zhang_Xian,Random_variables,Dimensional_theory,Zhang_Mengjie,Xie_Shenghan,Geng_Xianjin},并在纯数学和自然科学等多个领域中得到了比较多的成果.目前对连分数的研究主要涉及到动力系统,遍历论,概率论,随机过程,分形几何等知识.
在本文中,我们将给出连分数收敛的判定定理;解释连分数部分商序列严平稳,可逆等性质;介绍混合性与遍历性的基本知识;并证明连分数动力系统的强混合性;由强混合性与遍历性的相关关系证明连分数动力系统的遍历性,并进行简单运用.\par
本论文概要如下:
\begin{enumerate}
      \item 绪论,主要介绍连分数的研究历史和现状,并介绍文章主要结论.
      \item 对连分数的定义、收敛性、度量理论以及Gauss变换和Gauss测度以及部分商序列的一些性质进行介绍.
      \item 对动力系统混合性的介绍,证明Kuzmin定理,并由此解决Gauss问题及证明连分数动力系统的强混合性,连分数部分商序列的$\psi$-混合性.
      \item 对动力系统遍历性的介绍,由连分数动力系统的强混合性证明连分数动力系统的遍历性,并给出由遍历性得到的关于连分数部分商序列的结论.
\end{enumerate}

\sectionbreak